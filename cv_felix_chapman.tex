% LaTeX file for resume
% This file uses the resume document class (res.cls)

\documentclass[margin]{res}
\usepackage{hyperref}

%\usepackage{helvetica} % uses helvetica postscript font (download helvetica.sty)
%\usepackage{newcent}   % uses new century schoolbook postscript font
\topmargin=-0.5in  % start text higher on the page
\textheight=700pt % add more space to page

\begin{document}
\name{FELIX CHAPMAN \vspace{0.2in}}

\address{
\begin{minipage}[t]{0.25\textwidth}
London \\
United Kingdom
\end{minipage}
\begin{minipage}[t]{0.25\textwidth}
+44 7977 871939 \\
\href{https://ael.red/}{https://ael.red/} \\
\href{mailto:aelred717@gmail.com}{aelred717@gmail.com} \\
\href{https://www.github.com/aelred}{github.com/aelred}
\vspace{0.1in} \\
\end{minipage}
}

\begin{resume}

\section{PERSONAL STATEMENT}

Software engineer with 7+ years of professional experience in backend development primarily in Java, Scala, Python and Rust.

\section{EXPERIENCE}

\normalsize{\section{2023}}
{\bf Babylon Health} (\href{https://www.babylonhealth.com/}{https://www.babylonhealth.com}) \\
Senior Software Engineer

\begin{itemize}

\item
Designed and developed data ingestion pipelines using Python to extract partner's healthcare data, transform it into our internal standardised format and load it into our Health Graph, enabling crucial analytics use cases.

\end{itemize}

\normalsize{\section{2019 - 2023}}
{\bf Babylon Health} (\href{https://www.babylonhealth.com/}{https://www.babylonhealth.com}) \\
Software Engineer
\begin{itemize}

\item
Developed Health Graph, a unified view of our healthcare data, which reads events from Kafka, writes them to Elasticsearch and exposes them as BigQuery tables and in a standard healthcare data format (FHIR) with a GraphQL API.

\item
Supported several teams to integrate with Health Graph, coaching developers and clinicians on how to write schemas and how to publish healthcare data events until they were able to do this themselves.

\item
Designed and developed a "self-serve" approach for teams to publish new data types to Health Graph, including publishing new schemas which are automatically pushed to S3, transformed into protobuf and made available in Health Graph.

\item
Created What the Fox, an internal Slack tool written in Scala for looking up acronyms and jargon. Anyone can contribute terms and the tool tracks frequently requested terms.

\end{itemize}

\normalsize{\section{2015 - 2018}}
{\bf Vaticle} (formerly Grakn Labs, \href{https://typedb.com/}{https://typedb.com/}) \\
Core Developer (Software Engineer)
\begin{itemize}

\item
Created TypeQL (formerly Graql, \href{https://typedb.com/docs/typeql/2.x/overview.html}{https://typedb.com/docs/typeql/2.x/overview.html}), a declarative pattern-based query language that operates on the TypeDB (formerly Grakn) graph database, implemented in Java, using Tinkerpop and ANTLR.

\item
Achieved remote query execution using protocol buffers and gRPC to stream results over HTTP. I implemented remote clients in Java, Python and Haskell.

\item
Improved performance of query execution in Moogi - a movie semantic search engine used to demonstrate our software stack.

\item
Supported loading database schemas into the graph by building a parser for CTM (compact topic map) files.

\item
Helped implement company processes such as code reviews and automated testing using Travis CI and Jenkins.

\end{itemize}

\normalsize{\section{2014}}
{\bf University of Southampton} \\
Internship, PLEASED project

\begin{itemize}

\item
Classified signals from plants in response to stimulus with Python.

\end{itemize}

\pagebreak

\section{VOLUNTEERING}

\normalsize{\section{2018}}
{\bf Montana Youth Hostel, Bergen} (\href{https://www.montana.no/}{https://www.montana.no/}) \\
Social Volunteer

\section{PROJECTS}

\normalsize{\section{Skakoui}}
A traditional chess bot, that competes online (\href{https://lichess.org/@/skakoui}{https://lichess.org/@/skakoui})

\normalsize{\section{NES emulator}}
A NES emulator written in Rust. (\href{https://github.com/aelred/nes-rust}{https://github.com/aelred/nes-rust})

\normalsize{\section{Tetris}}
Project to learn Rust. Used SDL and compiled into Wasm. (\href{https://tetris.ael.red/}{https://tetris.ael.red/})

\normalsize{\section{Starling}}
A functional programming language implemented in Python. (\href{https://github.com/aelred/starling}{https://github.com/aelred/starling})

\section{EDUCATION}
\normalsize{\section{2011 - 2015}}
{\bf University of Southampton} \\
MEng Computer Science with First Class Honours \\

\section{SKILLS}

\normalsize{\section{Programming languages}}
Proficient in Java, Scala, Python and Rust. \\
Experience with JavaScript, TypeScript, C\#, C, Kotlin, Haskell and Visual Basic.

\normalsize{\section{Technologies}}
gRPC, protocol buffers, Git, ANTLR, SDL, Tinkerpop, Gradle, Maven, CircleCI, Jenkins, FHIR, Terraform, Kubernetes

\end{resume}

\end{document}

